\chapter{Processo di sviluppo}
Il processo di sviluppo utilizzato per la realizzazione del software è \textbf{UP} (Unified Process). Un processo di tipo agile ci ha permesso quindi in brevi tempi di avere un'implementazione funzionante del software, anche se non soddisfa tutti i casi d'uso richiesti.
\\Data la grande dimensione dell'applicazione e i tempi ridotti di consegna, l'utilizzo di un processo \textit{guidato da piani} avrebbe generato non pochi problemi.

\section{Github}
In un primo approccio alla realizzazione del sistema si è deciso di utilizzare \textbf{Github} come piattaforma di condivisione di codice, modello e documentazione. Fin dall'inizio il lavoro è stato diviso in tre repository indipendenti:
\begin{itemize}
	\item Una repository che contiene tutti i file di documentazione e di modellazione del software.
	\item Una repository che contiene solo la parte di software \textit{backend}. 
	\item Una repository che contiene solo la parte di software \textit{frontend}.
\end{itemize}
Il motivo della divisione del codice in più repository è quello di scollegare fisicamente la realizzazione delle due parti. Inoltre il \textit{frontend} è pensato per essere implementato su una piattaforma Android, mentre il \textit{backend} su una piattaforma Desktop, accentuando tale suddivisione.
\\In ogni repository sono state create delle \textit{Branch} per permettere uno sviluppo parallelo tra i vari componenti del gruppo. Infatti durante la realizzazione ogni componente si è occupato di un lavoro, al termine dei quali si poteva continuare con un \textit{Merge} per unirli.
\\All’inizio del lavoro si è anche fatto uso delle fork.
In particolare si sono usate due diverse fork per il lato backend.
La prima fork riguardava lo sviluppo dei proxy e la seconda lo sviluppo dell’applicazione centrale.
In seguito le due fork sono state unite durante l’integrazione.

\section{UP}
Il processo di sviluppo software UP prevede lo sviluppo in iterazioni. Ognuna delle quali comprende \textit{analisi}, \textit{progettazione}, \textit{implementazione} e \textit{test}.
\\Prima di tutto si è effettuata una fase di \textbf{ideazione}, per poi proseguire con la fase di \textbf{elaborazione} (che comprende le iterazioni vere e proprie) ed infine una fase di \textbf{transazione} per rilasciare una piccola versione ed illustrarne il funzionamento.

\subsection{Ideazione}
Nei primi giorni si è deciso \textit{cosa} il sistema dovesse fare. In particolare sono stati identificati gli attori che avrebbero preso parte all'applicativo e per ogni attore gli obiettivi da raggiungere. Sono stati valutati i requisiti considerando anche il tempo che si aveva a disposizione per non sfociare in un progetto troppo complesso da realizzare.
\\In seguito si è effettuato uno studio di fattibilità, volto a capire se il sistema poteva essere realizzato con gli strumenti che si avevano a disposizione. 
\\Nel dettaglio nella fase di ideazione sono state eseguite le seguenti azioni:
\begin{itemize}
	\item Workshop dei requisiti, attraverso il confronto con gli stackholder (alcuni componenti del gruppo hanno lavorato e/o lavorano ancora nell'ambito della ristorazione).
	\item Identificazione degli attori.
	\item Descrizione breve dei primi casi d'uso ricercati.
	\item Descrizione dettagliata dei casi d'uso più importanti, quelli che rappresentavano il "cuore" del sistema e che racchiudevano le funzionalità di principale interesse.
	\item Architettura di alto livello 	
\end{itemize} 

\subsection{Elaborazione}
La fase di elaborazione è stata divisa in tre iterazioni di all'incirca 1-2 settimane ciascuna.
\begin{enumerate}
	\item Nella prima iterazione sono stati analizzati i requisti più nello specifico focalizzando l'attenzione sulla logica centrale del sistema, costruendo:
	\begin{itemize}
		\item Analisi
		\begin{itemize}
			\item Modello di dominio
			\item Sequence Diagram di sistema
			\item Comunicazione tra Business Logic e Database
		\end{itemize}
		\item Progetto
		\begin{itemize}
			\item Progetto dell'architettura del sistema
			\item Progetto della base di dati
			\item Progetto dell'architettura del Main System
			\item Class Diagram di dettaglio della Business Logic
			\item Sequence Diagram di dettaglio
		\end{itemize}
		\item Implementazione
		\begin{itemize}
			\item Implementazione Business Logic (BL)
			\item Implementazione del database (DB)
		\end{itemize}
		\item Test
		\begin{itemize}
			\item Test di unità della BL
			\item Test di integrazione tra BL e DB
		\end{itemize}
	\end{itemize}

	\item Nella seconda iterazione:
	\begin{itemize}
		\item Analisi
		\begin{itemize}
			\item Comunicazione tra Business Logic, Broker e Request Generator
			\item Comunicazione tra Proxy e Main System
		\end{itemize}
		\item Progetto
		\begin{itemize}
			\item Class diagram di dettaglio del Broker
			\item Class diagram di dettaglio del Request Generator
			\item Class diagram di dettaglio dei Proxy
		\end{itemize}
		\item Implementazione
		\begin{itemize}
			\item Implementazione del Broker
			\item Implementazione del Request Generator
			\item Implementazione dei Proxy
		\end{itemize}
		\item Test
		\begin{itemize}
			\item Test di integrazione tra Proxy, Request Generator e Broker
		\end{itemize}
	\end{itemize}
	\item Nella terza iterazione ci si è focalizzati sul lato \textit{frontend} per rilasciare un applicativo accessibile dall'utente.
		\begin{itemize}
			\item Analisi
			\begin{itemize}
				\item Comunicazione tra Proxy e Applicazione Android
				\item Formato dei messaggi da scambiare
			\end{itemize}
			\item Progetto
			\begin{itemize}
				\item Progetto dell'architettura Android
			\end{itemize}
			\item Implementazione
			\begin{itemize}
				\item Raffinamento Main System 
				\item Implementazione dell'applicazione Android
			\end{itemize}
			\item Test
			\begin{itemize}
				\item Test di comunicazione tra Android e Sistema
			\end{itemize}
		\end{itemize}
\end{enumerate}

\vspace{1cm}
Le fasi Costruzione e Transazione non sono state applicate a causa delle tempistiche, anche se sono previste secondo il modello UP. Al termine della fase di elaborazione però viene rilasciato una piccola demo per l'applicazione Android, con compatibilità minima del sistema operativo \textit{Lollipop 5.0}, per testare il corretto funzionamento dei casi d'uso implementati.

\section{Altre pratiche}
Vista la natura fortemente disaccoppiata delle componenti ed il numero delle persone presenti nel team si è potuto andare a dividere facilmente il lavoro.
In particolare nelle fasi iniziali si è usato il pair programming per il sistema centrale e per i proxy.
\begin{itemize}
	\item Antonio Emmanuele e Giuseppe Di Cecio si sono occupati insieme del sistema centrale.
	\item Giuseppe De Rosa e Nicola D’Ambra si sono occupati dello sviluppo dei Proxy.
\end{itemize}
In seguito, quando si è passati alla necessità di integrazione, si è continuato ad usare il pair programming per integrare proxy ed applicazione centrale. 
\begin{itemize}
	\item Antonio Emmanuele e Giuseppe De Rosa si sono occupati dell'integrazione tra Proxy e Main System.
	\item Giuseppe Di Cecio e Nicola D’Ambra si sono occupati dello sviluppo dell'applicazione Android.
\end{itemize}
Altra pratica che abbiamo integrato è quella, presa dal framework scrum, del daily scrum.
Ogni giorno c’è infatti stata una riunione in cui ci si è aggiornati sullo stato di tutti i lavori.
Ovviamente, avendo usato il pair programming, il \textbf{daily scrum} si è svolto tra i due gruppi e non per ogni singola persona del gruppo.