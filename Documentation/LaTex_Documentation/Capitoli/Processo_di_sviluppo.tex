\chapter{Processo di sviluppo}
Il processo di sviluppo utilizzato per la realizzazione (parziale) del software è \textbf{UP} (Unified Process). Un processo di tipo agile ci ha permesso quindi in brevi tempi avere un'implementazione funzionante del software, anche se non soddisfa tutti i casi d'uso richiesti.
\\Data la grande dimensione dell'applicazione e i tempi ridotti di consegna, l'utilizzo di un processo \textit{guidato da piani} avrebbe generato non pochi problemi.

\section{Github}
In primo approccio alla realizzazione del sistema si è deciso di utilizzare \textbf{Github} come piattaforma di condivisione di codice, modello e documentazione. Fin dall'inizio il lavoro è stato diviso in tre repository indipendenti:
\begin{itemize}
	\item Una repository che contiene tutti i file di documentazione e di modellazione del software.
	\item Una repository che contiene solo la parte di software \textit{backend}. 
	\item Una repository che contiene solo la parte di software \textit{frontend}.
\end{itemize}
Il motivo della divisione del codice in più repository è quello di scollegare fisicamente la realizzazione delle due parti. Inoltre il \textit{frontend} è pensato per essere implementato su una piattaforma Android, mentre il \textit{backend} su una piattaforma Desktop, accentuando tale suddivisione.
\\In ogni repository sono state create delle \textit{Branch} per permettere uno sviluppo parallelo tra i vari componenti del gruppo. Infatti durante la realizzazione ogni componente si è occupato di un lavoro, al termine dei quali si poteva continuare con un \textit{Merge} per unirli.

\section{UP}
Il processo di sviluppo software UP preve lo sviluppo in iterazioni. Ognuna delle quali comprende \textit{analisi}, \textit{progettazione}, \textit{implementazione} e \textit{test}.
\\Prima di tutto si è effettuata una fase di \textbf{ideazione}, per poi proseguire con la fase di \textit{elaborazione} (che comprende le iterazioni vere e proprie) ed infine una fase di \textit{transazione} per rilasciare una piccola versione ed illustrarne il funzionamento.

\subsection{Ideazione}
Nei primi giorni si è deciso \textit{cosa} il sistema dovesse fare. In particolare sono stati identificati gli attori che avrebbero preso parte all'applicativo e per ogni attore gli obiettivi da raggiungere. Sono stati valutati i requisiti considerando anche il tempo che si aveva a disposizione per non sfociare in un progetto troppo complesso da realizzare.
\\In seguito si è effettuato uno studio di fattibilità. Capire se il sistema poteva essere realizzato con gli strumenti che si avevano a disposizione. 
\\Nel dettaglio nella fase di ideazione sono state eseguite le seguenti azioni:
\begin{itemize}
	\item Workshop dei requisiti, attraverso il confronto con gli stackholder (alcuni componenti del gruppo hanno lavorato e/o lavorano ancora nell'ambito della ristorazione).
	\item Identificazione degli attori.
	\item Descrizione breve dei primi casi d'uso ricercati.
	\item Descrizione dettagliata dei casi d'uso più importanti, quelli che rappresentavano il "cuore" del sistema e che racchiudevano le funzionalità di principale interesse.
	\item Architettura di alto livello 	
\end{itemize} 

\subsection{Elaborazione}
La fase di elaborazione è stata divisa in due iterazioni di all'incirca 2-3 settimane ciascuna.
\begin{enumerate}
	\item Nella prima iterazione sono stati analizzati i requisti più nello specifico, costruendo:
	\begin{itemize}
		\item Modello di dominio
		\item Sequence Diagram di sistema, per i casi d'uso scelti per l'implementazione
		\item Progetto dell'architettura del sistema
		\item Progetto della base di dati
		\item Implementazione parziale del software \textit{backend}
	\end{itemize}
	\item Nella seconda iterazione puntava a realizzare l'applicazione \textit{frontend} per avere un primo rilascio del sistema. In particolare:
	\begin{itemize}
		\item Class diagram di dettaglio
		\item Progettazione della comunicazione tra  \textit{backend} e \textit{frontend}
		\item Definizione dei messaggi da scambiare
		\item Progetto dell'archietettura Android
		\item Implementazione completa del software
	\end{itemize}
\end{enumerate}