\chapter{Implementazione}
L’implementazione è avvenuta usando il framework \textbf{Spring} vista la sua semplicità di utilizzo e la velocità di implementazione di un server per i test.
Altro vantaggio di questo framework è l’enorme quantità di documentazione e di discussioni reperibili in rete che rendono ogni problema di facile risoluzione.
\\L’enorme supporto di Spring ha inoltre concesso una libera scelta per le altre tecnologie usate nel progetto. In particolare ci ha concesso di usare ed integrare facilmente un database \textbf{PostgreSQL} ed \textbf{ActiveMQ Artemis} per la comunicazione tra Main System e Proxy.
\vspace{0.5cm}
PostgreSQL è stato scelto per il suo essere open source e per la sua facilità di utilizzo. Ciononostante si fa presente che grazie alla \textit{Dependency Injection} di Spring sarebbe facile cambiare database, pertanto la sua scelta non è stata dirimente nel progetto.
\vspace{0.5cm}
Artemis ActiveMQ è stato scelto come servizio di messaggistica in quanto:
\begin{itemize}
	\item Implementa il JMS 2.0
	\item E’ un MOM configurabile in maniera particolarmente rapida: basta usare un comando per dichiararsi un broker ed un altro per avviarlo e subito si può iniziare a sperimentare.
\end{itemize}
