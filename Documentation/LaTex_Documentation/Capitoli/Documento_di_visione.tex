\chapter{Documento di visione}
\setstretch{2}
\section{Introduzione}
\setstretch{1}
\justifying

Si vuole realizzare un sistema per la gestione degli ordini e dei pagamenti di un locale di ristorazione.\\
L’obiettivo è automatizzare e semplificare la gestione delle attività di routine del personale, a partire dalle ordinazioni dei clienti fino ad arrivare all’organizzazione della cucina, della sala e dei pagamenti.\\ 
In particolare si vuole far fronte a tutte quelle problematiche legate alle :
\begin{itemize}
    \item Ordinazioni dei clienti.
    \item Amministrazione della sala.
    \item Sincronizzazione tra le varie attività all’interno del locale (cucina, bar, forno, etc.).
    \item Gestione delle merci a disposizione.
    \item Pagamenti dei clienti a fine servizio.
    \item Storici dei dati del locale.

\end{itemize}
\hfill \break

\section{Parti interessate}
\begin{enumerate}
    \item Proprietario del locale.
    \item Dipendenti del locale (camerieri, pizzaioli, chef, etc.).
    \item Clienti del locale.
\end{enumerate}

\section{Utenti}
\begin{itemize}
    \item Dipendti di sala (camerieri, addetto all'accoglienza) 
    \item Dipendenti di cucina/forno (chef, pizzaiolo).
    \item Proprietario del locale.
    \item Gestore delle merci (economo).
\end{itemize}

\section{Obiettivi delle parti interessate}
\begin{enumerate}
    \item Un servizio più efficiente e automatizzato massimizza il numero di clienti serviti e la loro soddisfazione, minimizzando errori umani da parte dei dipendenti.
    \item Il lavoro del dipendente viene semplificato automatizzando la gestione delle ordinazioni al sistema.
    \item I clienti avranno un’esperienza migliore perché verranno ridotti i tempi di attesa e possibili errori dovuti da ordinazioni sbagliate. 
\end{enumerate}

\section{Caratteristiche del sistema}

Il sistema sarà in esecuzione su un’applicazione direttamente accessibile dai dipendenti del locale. \\
In primo luogo il dipendente della sala dovrà essere in grado di registrare le ordinazioni dei clienti nel sistema il quale provvederà ad inviarle alle attività del locale interessate (cucina, forno o bar) . In secondo luogo il sistema dovrà essere accessibile dagli altri dipendenti, come dipendenti della cucina e del bar, per visualizzare le ordinazioni dei clienti registrate recentemente. \\Il sistema dovrà permette di gestire il coordinamento tra reparti diversi attraverso l'uso di messaggi di notifica ed inoltre dovrà fornire da supporto per l’assegnazione dei tavoli .
Infine grazie alla memorizzazione di tutte le transazioni il proprietario potrà conoscere i guadagni del locale, la merce consumata e ricavarne statistiche.


