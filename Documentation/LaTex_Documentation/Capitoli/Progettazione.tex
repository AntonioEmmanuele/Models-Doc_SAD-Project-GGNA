\chapter{Progettazione}
La fase di progettazione prevede principalmente la scelta dell'architettura da usare, su cui costruire il sistema. 
\\Dato il contesto di utilizzo dell'applicativo, la necessità principale è la sincronizzazione del lavoro dei dipendenti. L'obiettivo è quello di informare ogni dipendente del avvenimento di eventi a cui esso è interessato. Ogni azione deve essere memorizzata nel sistema, in conseguenza della quale esso provvede ad avvisare il gruppo di dipendenti relativo.
\\Ad esempio, al momento della registrazione di un'ordinazione da parte di un cameriere, il sistema provvede a notificare lo chef/pizzaiolo (realizzatore in generale) per informalo della nuova pietanza da preparare.
\\Per identificare l'architettura ideale bisogna elencare quali sono i componenti che ne fanno parte e in che modo sono interconnessi.
\begin{itemize}
	\item I componenti sono di due tipologie:
		\begin{itemize}
			\item Dispositivi utente, rappresentano i dispositivi fisici concessi ai dipendenti (smartphone o applicazione desktop)
			\item Sistema centrale, racchiude la logica di gestione e a cui i dispositivi utente possono fare accesso.
		\end{itemize}
	\item I connettori sono principalmente protocolli di rete, in quanto il sistema è pensato anche per dipendenti che hanno la possibilità di spostarsi all'interno del locale e accedere al sistema centrale in modalità remota.
\end{itemize}
Alla base di ciò i componenti devono ricevere dei cambiamenti del sistema tramite opportune \textit{notifiche}, evitando quindi attese attive.

\section{Architettura}
Sulla base delle necessità descritte in precedenza, si è scelto di utilizzare uno stile architetturale del tipo \textbf{Publish-Subscribe}. Per la realizzazione dell'applicativo infatti vengono ripresi tutti i vantaggi dello stile a \textit{invocazione implicita}:
\begin{itemize}
	\item \textit{Disaccoppiamento spaziale}: tutti i componenti sono altamente disaccoppiati, favorendo la scalabilità del sistema. Il proprietario può quindi assumere un numero di dipendenti variabile senza conseguenze.
	\item \textit{Disaccoppiamento di sincronizzazione}: tutti i componenti non lavorano in "polling", ovvero in attese attive che possono bloccare il sistema.
	\item \textit{Disaccoppiamento temporale}: tutti i componenti possono essere volatili nel sistema.
\end{itemize}
In relazione ai componenti descritti, tutti i dipendenti sono dei \textit{Subscribers} mentre il sistema centrale è l'unico \textit{Publisher}.
\\I dipendenti, effettuando l'accesso, si iscrivono in automatico al Publisher.
Dato che il sistema può contenere un numero notevole di dipendenti, il Publisher fa uso di \textbf{Proxy} per smistare i messaggi solamente a specifici Subscribers.
\\Essi infatti all'atto del login non conoscono i Proxy del sistema, ma è il sistema stesso che li racchiude in gruppi (con lo stesso ruolo) e li associa a determinati Proxy. Tutto ciò si traduce in una distribuzione più efficiente del calcolo computazionale.
\\Riassumendo quindi:
\begin{table}[H]
	\centering
	\begin{tabular}{|l | p{0.75\linewidth} |}
		\hline
		\textbf{Componenti} & Publisher, Subscribers, 
		Proxy \\
		\hline
		\textbf{Connettori} & Protocolli di rete \\
		\hline
		\textbf{Dati} & Notifiche, Richieste di informazioni, Informazioni pubblicate \\
		\hline
		\textbf{Topologia} & Subscribers sono connessi al Publisher indirettamente, ricevendo e inviando notifiche attraverso i Proxy \\
		\hline
		\textbf{Vantaggi} & Subscribers sono tutti disaccoppiati lavorando in parallelo. Le notifiche vengono ben distribuite grazie ai Proxy \\
		\hline
		\textbf{Svantaggi} & Il controllo computazionale è caricato tutto sul sistema centrale. Nessuna conoscenza di quali componenti risponderanno ad un eventi \\
		\hline
	\end{tabular}
\end{table}

\subsection{Main System}
Il sistema principale è l'unico Publisher dello stile utilizzato. Studiando i vari stili architetturali, tra quelli visti durante il corso, e prendendo spunto dallo stile a livelli, l'architettura del sistema non si rispecchia in uno stile architetturale ben preciso, ma è caratterizzata nel seguente modo:
\begin{figure}[H]
	\centering
	\includegraphics[width=0.4\textwidth]{Immagini/main_system.png}
\end{figure}
\begin{itemize}
	\item \textbf{Database}: Al livello più basso, il componente, racchiude tutte le funzionalità per accedere e modificare i dati all'interno della base di dati del sistema. La business logic accede al database solo attraverso all'interfaccia che tale livello offre, mascherando la reale implementazione.
	\item \textbf{Business Logic}: Al livello centrale, il componente, racchiude tutte le funzionalità per manipolare le richieste e i dati del sistema. Effettua tutti i controlli necessari prima di inviare una risposta. Si occupa principalmente di implementare i casi d'uso.
	\item \textbf{Broker}: Al livello più alto il Broker si occupa solo di inviare i le risposte (prodotte dalla Business Logic) ai diretti interessati. 
	\item \textbf{Request Generator}: Al livello più alto esso riceve, e identifica, le richieste dall'esterno e le inoltra alla Business Logic che si occupa di elaborarle e inviare le risposte al Broker.
\end{itemize}
La Business Logic deve esporre un'interfaccia al Request Generator che contiene le funzioni in relazione al caso d'uso da realizzare.
\\Tutto il componente viene visto dall'esterno (e quindi dai Proxy) come una blackbox con due sole interfacce: una per ricevere delle richieste e una per pubblicare gli eventi.

\subsection{Proxy}
Il proxy contiene un'insieme di utenti connessi al sistema tramite applicazione. Esistono più proxy, ognuno associato ad una categoria di utenti (i.e. Proxy Camerieri è associato a tutti i camerieri). Così facendo, all'atto dell'invio di un messaggio il Broker lo invia ad un relativo Proxy che provvede a smistarlo correttamente. 
\begin{figure}[H]
	\centering
	\includegraphics[width=0.6\textwidth]{Immagini/proxy.png}
\end{figure}
Oltre a inviare i messaggi a tutti i dispositivi associati, esso riceve anche informazioni solo da essi.
\\Esso è quindi dotato di tre interfacce:
\begin{itemize}
	\item Un'interfaccia per collegarsi al Broker. Il Broker invia un evento da pubblicare, il proxy lo pubblica a tutti i suoi dispositivi.
	\item Un'interfaccia per collegarsi al Request Generator. Il proxy riceve dai dispositivi eventi di richiesta, il Request Generator riceve da esso l'evento da elaborare.
	\item Un'interfaccia per collegarsi all'applicazione mobile.
\end{itemize}
Attraverso l'analisi degli attori precedentemente descritta si hanno sei tipi di proxy:
\begin{itemize}
	\item Proxy Cameriere associato a tutti i camerieri in servizio.
	\item Proxy Realizzatori associato a tutti i realizzatori di ordinazioni (chef, pizzaiolo, barista).
	\item Proxy Accoglienza associato all'addetto all'accoglienza.
	\item Proxy Cossiere associato al cassiere in servizio.
	\item Proxy Gestione associato al proprietario e all'economo.
	\item Proxy Login è un proxy generale a cui vengono associati tutti gli utenti che non hanno ancora effettuato l'accesso.
\end{itemize}
L'idea di base è che ogni utente, prima di effettuare l'accesso, può contattare solo il \textit{Proxy Login}, il quale provvede in base alle credenziali ad associare il dispositivo al relativo Proxy.

\subsection{Applicazione Mobile}
Tutto il calcolo computazionale è a carico del \textit{Main System}, mentre lo smistamento dei messaggi è a carico dei \textit{Proxy}. L'applicazione deve solo ricevere e inviare eventi dal proxy (o dai proxy, se più di uno) associato. La costruzione del messaggio avviene tramite un'interfaccia grafica per ogni contesto, il resto è gestito tutto dal Main System.
\\A tal proposito si è utilizzato il pattern Architetturale \textit{MVC (Model View Control)} per la realizzazione dell'applicazione.
\begin{itemize}
	\item \textit{Modello} contiene tutti i dati necessari per il messaggio da inviare o ricevuto dall'esterno.
	\item \textit{Controllo} contiene tutte le funzionalità per interfacciarsi all'esterno tramite opportuni protocolli di rete per inviare e ricevere messaggi.
	\item \textit{View} è la rappresentazione grafica dei dati.
\end{itemize}



\subsection{Component Diagram}
\begin{figure}[H]
	\centering
	\includegraphics[width=1\textwidth]{Immagini/components.jpg}
\end{figure}
Per rendere più chiara la lettura del diagramma, i proxy sono stati \underline{replicati} solo ad uno scopo grafico. 
\\L'applicazione inoltre non è collegata realmente a tutti i proxy ma solo a quelli a cui corrispondono le sue credenziali di accesso; potenzialmente può essere collegata a tutti i proxy.

\section{Class Diagram}
Dopo aver illustrato il component diagram, si deve illustrare il modello ad oggetti di ogni componente presente nel sistema.


\subsection{Main System}
Il Main System è il componente più complesso perché contiene a sua volta altri componenti connessi tra loro, quindi non può essere rappresentato con un unico class diagram. 
\\Nel livello inferiore è presente il database. Esso non può essere rappresentato come class diagram per la sua natura, inoltre non deve essere realizzato poiché ci si affida a database \textit{relazionali} già esistenti.

\subsubsection{Business Logic}
Il class diagram è diviso in layer orizzontali, in cui ogni livello può interfacciarsi solo con il livello inferiore e può offrire un'interfaccia al livello superiore:
\begin{itemize}
	\item \textbf{DataAccess}: livello più basso si occupa di implementare il driver per l'accesso al database.
	\item \textbf{Areas}: livello centrale mantiene il modello ad oggetti del sistema vero e proprio.
	\item \textbf{Controller}: livello più alto può accedere al modello inferiore e fornire un'interfaccia per un accesso controllato ai dati.
\end{itemize}
\begin{figure}[H]
	\centering
	\includegraphics[width=0.6\textwidth]{Immagini/business_logic.jpg}
\end{figure}
\hrule
\vspace{0.5cm}
\textbf{Areas}:
Il livello dei dati è a sua volta diviso in layer verticali, in cui ogni livello si occupa di gestire specifiche macro-responsabilita':
\begin{itemize}
	\item \textit{MenuAndWareHouseArea} si occupa di gestire le richieste di visualizzazione di prodotti sul menu, di merci, di aggiornamento della disponibilità dei prodotti in relazione alle merci ed alla creazione e personalizzazione di prodotti ordinati.
	\item \textit{TableAndOrdersArea} si occupa di gestire lo stato dei tavoli, la creazione degli ordini ad un tavolo e la modifica degli ordini.
	\item \textit{UserData} si occupa di verificare quali siano i ruoli di un utente fornendoli al richiedente, e di registrare gli ordini associati agli utenti.
\end{itemize}
Nel dettaglio i tre package sono progettati nel seguente modo.
\vspace{0.5cm}
\\\textit{MenuAndWareHouseArea}:
\begin{figure}[H]
	\centering
	\includegraphics[width=0.8\textwidth]{Immagini/MenuAndWareHouseArea.jpg}
\end{figure}
La classe \textit{orderedItem} in linea con lo state chart descritto nei capitoli precedenti è stato progettato utilizzando il \underline{design patter State} ed è associato a i prodotti del menu.
\vspace{0.5cm}
\\\textit{TableAndOrdersArea}:
\begin{figure}[H]
	\centering
	\includegraphics[width=0.8\textwidth]{Immagini/TableAndOrdersArea.jpg}
\end{figure}
Anche in questo caso in linea con lo state chart il \textit{Table} è stato progettato utilizzando il \underline{design patter State}.
\vspace{0.5cm}
\\\textit{UserData}:
\begin{figure}[H]
	\centering
	\includegraphics[width=0.3\textwidth]{Immagini/UserData.jpg}
\end{figure}
\vspace{0.5cm}
\hrule
\vspace{0.5cm}
\textbf{DataAccess}:
\begin{figure}[H]
	\centering
	\includegraphics[width=0.8\textwidth]{Immagini/DataAccess.jpg}
\end{figure}
Ad ogni layer verticale del livello superiore è associato una sola classe di accesso del livello inferiore. Così facendo ogni area può accedere solo alla parte di database interessata. 
\vspace{0.5cm}
\hrule
\vspace{0.5cm}
\textbf{Controller}:
\begin{figure}[H]
	\centering
	\includegraphics[width=0.8\textwidth]{Immagini/controller.jpg}
\end{figure}
Il \textit{SystemController} offre tutte le funzioni di callback, per prelevare le informazioni necessarie dal modello. Inoltre si serve dell'interfaccia del \textit{Broker} per pubblicare gli eventi necessari.

\subsubsection{Request Generator}
Si interfaccia con i \textit{Proxy} per ricevere i messaggi e, attraverso un Dispacher, selezionare la funzione di callback del \textit{Controller} giusta per la richiesta.
\begin{figure}[H]
	\centering
	\includegraphics[width=\textwidth]{Immagini/request_generator.jpg}
\end{figure}
La classe \textit{DispatcherInfo} a seconda del messaggio ricevuto chiama una funzione di callback dell'interfaccia di controller.
\\Il \textit{MessageListener} è l'interfaccia di un \textit{Receiver}. Esistono Receiver diversi, poiché i messaggi in arrivo hanno come sorgente Proxy diversi; identicamente però richiamano un'unica funzione del \textit{Dispatcher}.
\\Una seconda soluzione poteva essere di gestire un unico \textit{Receiver} aumentando però la complessità di implementazione. 	

\subsection{Broker}
Si interfaccia con i \textit{Proxy} per inviare i messaggi. Il broker come il duale del request generator.
\begin{figure}[H]
	\centering
	\includegraphics[width=0.6\textwidth]{Immagini/broker.jpg}
\end{figure}
Implementa l'interfaccia con i metodi di callback richiesti dal \textit{SystemController}, a seconda della richiesta viene richiamata uno specifico metodo del \textit{Sender}.
I metodi del \textit{Sender} possono essere di due tipi:
\begin{itemize}
	\item Info: inviano un evento(richiesta) solo a coloro che hanno generato l'evento scatenante.
	\item Notification: inoltrano una notifica. Una notifica è un evento generato dalla Business Logic per informare tutti i Subcribers della verifica di un evento.
\end{itemize}

\section{Database}
Il database deve contenere le informazioni di base che devono essere utilizzate, a partire dai dipendenti che vengono registrati nel sistema e finire con le ordinazioni che vengono effettuate durante l'utilizzo del sistema.
\begin{figure}[H]
	\centering
	\includegraphics[width=1\textwidth]{Immagini/database.jpg}
\end{figure}
Le caratteristiche principali del database sono le seguenti:
\begin{enumerate}
	\item Le merci hanno, oltre alle informazioni di base quali prezzo, codice identificativo, nome, etc, hanno un attributo \textit{priceAsAdditive}. Tale attributo, se non nullo, indica il prezzo da aggiungere se quella merce è usata come merce aggiuntiva di un prodotto.
	\item Il \textit{Prodotto Ordinato} è un'associazione ternaria tra \textit{Ordine}, \textit{Prodotto} e \textit{Merce}.
	 \item Ogni prodotto ha un attributo collegato all'entità \textit{Ruoli}. Esso serve per indicare a che attività appartiene quel prodotto.
\end{enumerate}
Il precedente database si riferisce solo ai casi d'uso che si sono scelti di implementare. Non contiene, ad esempio, la gestione delle vendite e lo storico del locale.

\section{Deployment Diagram}
\begin{figure}[H]
	\centering
	\includegraphics[width=1\textwidth]{Immagini/deploy.jpg}
\end{figure}
I Proxy sono progettati per comunicare tramite JMS quindi non richiedono che siano avviati nella stessa macchina. Essi possono essere collegati anche su altri \textit{device} e poi connessi in rete. Per motivi di mezzi a disposizione si è scelto di avviare i proxy sulla stessa macchina del Main System.
\\Analogo discorso vale per il database.

\section{Sequence Diagram}
I sequence digram mostrati successivamente sono diagrammi di dettaglio che non mostrano la realizzazione delle chiamate a funzioni di livello più basso. Per esse si rimanda al progetto in Visual Paradigm.
\subsection{Menu Request}
Rappresenta la richiesta di visualizzare tutto il menu del locale.
 \begin{figure}[H]
 	\centering
 	\includegraphics[width=1\textwidth]{Immagini/top_menuRequest.jpg}
 \end{figure}
\textit{SystemController} viene chiamato tramite la \textit{controllerIface}. Analogo per il broker con la \textit{brokerIface}. Qualora il ruolo non sia corretto viene comunque pubblicata una risposta con un messaggio di errore "roleFailed".

\subsection{Order Request}
Rappresenta la richiesta di tutti gli ordini, con la possibilità di selezionare tutti gli ordini di un area (cucina, forno, ecc).
 \begin{figure}[H]
	\centering
	\includegraphics[width=1\textwidth]{Immagini/top_orderRequest.jpg}
\end{figure}

\subsection{Table Request}
Rappresenta la richiesta di tutti i tavoli di una sala.
 \begin{figure}[H]
	\centering
	\includegraphics[width=1\textwidth]{Immagini/top_tableRequest.jpg}
\end{figure}