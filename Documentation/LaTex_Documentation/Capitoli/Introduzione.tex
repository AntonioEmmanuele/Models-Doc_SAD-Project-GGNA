\chapter{Introduzione}
Il seguente documento descrive il processo di realizzazione di un applicazione gestionale per un locale di ristorazione. 
\\Il sistema realizzato ha come obiettivo quello di semplificare il lavoro dei dipendenti e la gestione del locale da parte del proprietario. 
Le principali figure che traggono vantaggio dall'applicazione sono :
\begin{itemize}
	\item Camerieri : possono prendere le ordinazioni dei clienti in modo del tutto automatizzato sfruttando un dispositivo mobile;
	\item Proprietario : può controllare la sua attività in tempo reale e ricavare statistiche dai dati memorizzati (i.e. vendite, merce consumata etc.) .
	\item Addetto all'accoglienza : può gestire lo stato dei tavoli.
	\item Chef/Pizzaiolo : può visualizzare gli ordini da preparare e segnalare quando ha terminato il suo compito.
\end{itemize}
Pertanto, i camerieri disporranno di una applicazione da eseguire su smartphone/tablet mentre i restanti dipendenti del locale avranno un'applicazione desktop con interfacce diverse in base al ruolo che essi ricoprono.
In seguito saranno descritti il processo di sviluppo adottato dal team, l'analisi e specifica dei requisiti, le scelte di progetto ed i dettagli implementativi .
